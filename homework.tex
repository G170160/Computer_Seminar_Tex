\documentclass[a4paper,12pt]{article}
\usepackage{booktabs}
\usepackage{amsmath}

\title{コンピューターゼミ2018宿題}

\begin{document}
\maketitle
\section{1章}
\par 私達の研究室ではおもにシステムやソフトウェアの信頼性に関する研究を行っています。
おもにそれらを確率論によってモデル化し、解析することで信頼性の評価を行います。
\par    具体的に以下の様な確率過程を用いることが多いです。
\begin{itemize}
    \item NHPP
    \item CTMC
\end{itemize}

\section{2章}
\par 卒業論文や現行の作成のさいには \LaTeX を使って文章を作成します。 \LaTeX は数式
などを含むような文章を綺麗に作成するための言語です。

\section{3章}
\par 確率変数$X$が指数分布に従うとき、その分布関数$F_x(t)$と密度関数$f_x(t)$は、
\begin{align}
F_x(t)&= 1-e^{-\lambda t} \\
f_x(t)&=\lambda e^{-\lambda t}
\end{align}
\par となる。またその期待値は定義より、
\begin{equation}
\begin{split}
E[X]&= \int_0^\infty tf_x(t)dt \\
&=[(1-e^{-\lambda t})t]^\infty_{0}-\int_0^{\infty}(1-e^{-\lambda t})dt \\
&=[(1-e^{-\lambda t})t]^\infty_{0}-[t+\dfrac{1}{\lambda}e^{-\lambda t})]^\infty_{0} \\
&=\dfrac{1}{\lambda}
\end{split}
\end{equation}
\par となる。(extra宿題:式(3)を導出してみよう ヒント:部分積分)

\section{4章}
\par 表をつくることもできます.


\centering
\begin{tabular}{|l|c|r|}
\hline
1 & 2 & 3 \\
\hline
$\alpha$ & $\beta$ & $\gamma$ \\
\hline
\end{tabular}
\end{document}